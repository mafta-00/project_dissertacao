O câncer é a segunda maior causa de morte em todo o mundo, sendo responsável por quase 10 milhões de mortes em 2020~\cite{ferlay2018global}.Esta doença começa com a transformação de células normais em células tumorais, num processo de múltiplas fases que geralmente progride de uma lesão pré-cancerosa para um tumor maligno. Diferentes partes do corpo humano podem ser afetadas por esta transformação. Nesse sentido, diversas pesquisas têm sido desenvolvidas com o objetivo de investigar como essas lesões acontecem em diferentes tipos de tecidos.

Uma dessas investigações está em desenvolvimento no Laboratório de Plasticidade Neural Entérica (LPNE) da Universidade Estadual de Maringá (UEM). Nesse trabalho, os pesquisadores têm avaliado as transformações provocadas pelo tumor Walker 256 nas células contidas em amostras de tecido hepático de ratos de laboratório, em cenário pré-clínico. Ao inspecionar visualmente essas imagens, eles notaram que diferentes padrões estão presentes quando amostras retiradas de indivíduos saudáveis e doentes são comparadas.

Os estudos realizados pelos pesquisadores do LPNE são feitos utilizando imagens capturadas por microscópios, as quais são analisadas de forma manual em um software proprietário. Tal software não é equipado com segmentadores automáticos capazes de realizar a segmentação dos núcleos para extração das propriedades geométricas como  área e perímetro e, também, não é possível realizar a contagem automática dos núcleos.  Por exemplo, para se obter a segmentação dos núcleos, é necessário que o pesquisador contorne o núcleo da célula manualmente, o que requer muito tempo e paciência. Já para realizar a contagem de núcleos que há na imagem é necessário utilizar uma grade, convencionada pelos pesquisadores do LPNE.

Com isso, é visto que existem dois problemas que são possíveis de serem solucionados utilizando as técnicas de Processamento Digital de Imagens (PDI). Problema 1 consiste em classificar uma imagem de tecido hepático de ratos e o Problema 2 consiste em segmentar e determinar a quantidade de núcleos que há em uma imagem de tecido hepático de ratos.

Diante destes problemas, com as dificuldades encontradas para segmentar e contar a quantidade de núcleos que há em uma imagem, se faz necessário o desenvolvimento de uma ferramenta capaz de segmentar e contar a quantidade de núcleos de forma automática em imagens de tecido hepático de ratos. Também se faz necessário desenvolver uma técnica para o problema de classificação.

Com isso, o objetivo geral desta dissertação é desenvolver um método para determinar a quantidade, a área e o perímetro de núcleo de células de imagens de tecidos hepáticos e desenvolver um classificador binário para separar as imagens em dois conjuntos, controle (C) e Tumor de Walker 256 (TW).

Este trabalho propõe dois métodos de segmentação para resolução do problema 2. O primeiro segmentador baseado nas técnicas de PDI, Morfologia Matemática (MM) e utilizando a rede neural convolucional, \textit{You Only Look Once} (YOLO). A YOLO será utilizada para encontrar os núcleos das células na imagem. Após isso, as regiões encontradas pela YOLO serão recortadas e a partir daí inicia-se a segmentação. Primeiro, são aplicadas as técnicas de PDI. Após isso, são aplicados às técnicas de MM para extração de ruídos e para melhorar os contornos dos núcleos segmentados. Já o segundo método de segmentação proposto utiliza somente técnicas de Morfologia Matemática para identificar os núcleos e segmentá-los e posteriormente técnicas de PDI para obter área e perímetro. 

A Morfologia Matemática é uma técnica de PDI que tem como objetivo extrair informações relativas à geometria e à topologia de um conjunto desconhecido de uma imagem utilizando-se dos conceitos matemáticos de teoria dos conjuntos \cite{pedrinischwartz}. A Morfologia Matemática é formada por duas operações básicas, dilatação e erosão, e as demais operações são combinações dessas duas operações básicas.

A rede neural YOLO é uma técnica de Visão Computacional publicada em 2016  por  Joseph Redmon. Esta técnica ganhou muita popularidade nos últimos anos, pelo seu desempenho na detecção de objetos em tempo real. A YOLO é capaz de identificar objetos em uma imagem em apenas 0,05.

Para o problema 1 são propostos dois métodos de obter os vetores de característica: descritor de textura LPQ (\textit{Local Phase Quantization}) e a técnica de Morfologia Matemática Granulometria. A classificação foi realizada usando três classificadores: k-NN, Regressão Logística e SVM. A escolha desses classificadores é justificada pelo tamanho do conjunto de dados, que é pequeno demais para alimentar modelos de aprendizagem profunda.

O descritor de textura LPQ apresenta um excelente desempenho em análise de textura de imagens. Sua principal característica é a robustez para imagens borradas ou afetadas por iluminação não uniforme. 

Este trabalho está dividido em cinco capítulos.

\begin{itemize}
	
	\item Capítulo 1: Introdução;
	
	\item Capítulo 2: Fundamentação teórica. Apresenta alguns conceitos introdutórios de imagem digital, Limiarização, Limiarização Otsu, Limiarização Adaptativa, os conceitos de Morfologia Matemática, descritor de textura LPQ, aprendizagem de máquina e, por fim, uma breve explicação da rede neural convolucional YOLO;
	
	\item Capítulo 3: Metodologia. Neste capítulo são descritos os métodos de classificação e segmentação desenvolvidos neste trabalho. Além disso, é apresentado o banco de dados que foi utilizado neste trabalho e, também, é definido os problemas do trabalho;
	
	\item Capítulo 4: Resultados. Este capítulo é dedicado à apresentação dos resultados obtidos para o problema 1 e para o problema 2.
	
	\item Capítulo 5: Conclusão. Por fim, neste capítulo são apresentadas as conclusões extraídas deste estudo.
	
\end{itemize}
